\mathchardef\mhyphen="2D 													% Define a "math hyphen"
\newcommand{\mathceil}[1]{\left \lceil #1 \right \rceil}					% ceil (rounding to smallest bigger integer)
\newcommand*\conj[1]{\mathop{\overline{#1}}\nolimits}						% conjugation
\newcommand{\orb}[1]{\langle #1 \rangle}									% Orbit
\newcommand{\cat}[1]{\mathscr{#1}}      									% Arbitrary category
\newcommand{\scat}[1]{\mathbf{#1}}      									% Arbitrary small category
\newcommand{\of}{\mathbin{\circ}}       									% Composition
\newcommand{\from}{{\colon}\linebreak[0]}       							% Colon for f: A --> B
\newcommand{\ot}{\leftarrow} 			  									% Left arrow
\newcommand{\toby}[1]{\xrightarrow{#1}}   									% Labelled arrow
\newcommand{\otby}[1]{\xleftarrow{#1}}   									% Labelled left arrow
\newcommand{\zero}[1]{\mathbf{\varnothing}_{#1}}							% Initial object
\newcommand{\one}[1]{\mathfrak{1}_{#1}}										% Terminal object
\newcommand{\toone}[1]{!_{#1}}												% Morphism to the terminal object
\newcommand{\comma}[2]{(#1 \mathbin{\downarrow} #2)}						% Comma category
\newcommand{\incl}{\hookrightarrow}            								% Inclusion arrow
\newcommand{\inclby}[1]{\xhookrightarrow{#1}}            					% Inclusion labeled arrow
\newcommand{\lcni}{\hookleftarrow}            								% Inclusion left arrow
\newcommand{\lcniby}[1]{\xhookleftarrow{#1}}            					% Inclusion left labeled arrow
\newcommand{\spn}[5]{#1 \otby{#2} #3 \toby{#4} #5}							% Span
\newcommand{\inclspn}[3]{#1 \lcni #2 \incl #3}								% Span with inclusions
\newcommand{\cospn}[5]{#1 \toby{#2} #3 \otby{#4} #5}						% Cospan
\newcommand{\inclcospn}[3]{#1 \incl #2 \lcni #3}							% Cospan with inclusions
\newcommand{\partialincl}[2]{#1 \lhook\joinrel\rightharpoonup #2}			% Partial monomorphism
\newcommand{\linearrule}[3]{#1 \lcniby{i_{#1}} #2 \inclby{i_{#3}} #3}		% Linear rule
\newcommand{\simplifiedrule}[2]{\linearrule{#1}{#1 \cap #2}{#2}}			% Simplified rule
\newcommand{\match}[3]{#1\from #2\incl #3}									% Monic match
\newcommand{\directderivation}[2]{\Rightarrow^{#1,#2}}						% Direct derivation
\newcommand{\lgraph}[1]{#1\mhyphen\linebreak[0]\scat{Graph}}				% Category of arc-labeled graphs
\newcommand{\sgraph}{\lgraph{\Sigma}}										% Category of graphs arc-labeled on \Sigma
\newcommand{\dims}{\mathbb{D}}												% Set of dimensions
\newcommand{\plustwo}[1]{#1_{+2}}											% Set of pairs (i,j) s.t. i+2<=j
\newcommand{\words}{\mathbb{W}}												% Set of words
\newcommand{\intset}[2]{#1..#2}												% Set of integers betwen two bounds
\newcommand{\qer}[2]{#1/\mathord{#2}}   									% Quotient by equivalence relation
\newcommand{\pathto}{\rightsquigarrow}										% Path
\newcommand{\lpathto}[1]{\overset{#1}{\rightsquigarrow}}					% Labeled path
\newcommand{\relabelingset}{\mathfrak{R}}									% Relabeling set
\newcommand{\embeddingfunctor}[1]{\mathbb{E}_{#1}}							% Embedding functor
\newcommand{\embeddingfunctordef}[3]{%										% Definition of the embedding functor
	\embeddingfunctor{#1}\from\lgraph{#2}\to\lgraph{#3}%
}
\newcommand{\wordsebdfunctor}{\embeddingfunctor{\dims}}						% Embedding functor for words
\newcommand{\wordsebdfunctordef}{%											% Definition of the embedding functor for words
	\embeddingfunctordef{\dims}{\words}{(\words \times \dims)}%
}
\newcommand{\projectingfunctor}[1]{\pi_{#1}}								% Projecting functor
\newcommand{\projectingfunctordef}[3]{%										% Definition of the projecting functor
	\projectingfunctor{#1}\from\lgraph{#2}\to\lgraph{#3}%
}
\newcommand{\dimsprojfunctor}{\projectingfunctor{\dims}}					% Projecting functor on dimensions
\newcommand{\dimsprojfunctordef}{%											% Definition of the projecting functor on dimensions
	\projectingfunctordef{\dims}{(\words \times \dims)}{\dims}%
}					
\newcommand{\wordsprojfunctor}{\projectingfunctor{\words}}					% Projecting functor on words
\newcommand{\wordsprojfunctordef}{%											% Definition of the projecting functor on words
	\projectingfunctordef{\dims}{(\words \times \dims)}{\words}%
}