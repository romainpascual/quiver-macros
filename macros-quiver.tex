%%%%%%%%%%%%%%%%%%%%%%%%%%%%%%%%%%%%%%%%
%           Commandes perso            %
%%%%%%%%%%%%%%%%%%%%%%%%%%%%%%%%%%%%%%%%

% References to notations and definitions
\newcommand{\idef}[3]{\emph{#1}\index{#3}\label{idef:#2}}

% Definitions and Theorems
\newtheorem{Def}{Definition}
\newtheorem{Prop}{Proposition}
\newtheorem{Lma}{Lemma}
\newtheorem{Thm}{Theorem}
\newtheorem{Fact}{Fact}
\newtheorem{Hyp}{Assumption}
\newtheorem{Ex}{Example}
\newtheorem*{Nota}{Notation}

\newcommand{\proofparagraph}[1]{%
	\paragraph{%
		\normalfont
		\(\blacktriangleright\)
		\emph{%
			#1
		}~\\\noindent%
	}%
}
\newcommand{\impliesproof}{\paragraph{(\(\Rightarrow\))}}
\newcommand{\impliedproof}{\paragraph{(\(\Leftarrow\))}}
\newcommand{\conclusionproof}{\bigskip \noindent}

% Maths
\mathchardef\mhyphen="2D 													% Define a "math hyphen"
\newcommand{\mathceil}[1]{\left \lceil #1 \right \rceil}					% ceil (rounding to smallest bigger integer)
\newcommand*\conj[1]{\mathop{\overline{#1}}\nolimits}						% conjugation
\newcommand{\orb}[1]{\langle #1 \rangle}									% Orbit
\newcommand{\cat}[1]{\mathscr{#1}}      									% Arbitrary category
\newcommand{\scat}[1]{\mathbf{#1}}      									% Arbitrary small category
\newcommand{\of}{\mathbin{\circ}}       									% Composition
\newcommand{\from}{{\colon}\linebreak[0]}       							% Colon for f: A --> B
\newcommand{\ot}{\leftarrow} 			  									% Left arrow
\newcommand{\toby}[1]{\xrightarrow{#1}}   									% Labelled arrow
\newcommand{\otby}[1]{\xleftarrow{#1}}   									% Labelled left arrow
\newcommand{\zero}[1]{\mathbf{\varnothing}_{#1}}							% Initial object
\newcommand{\one}[1]{\mathfrak{1}_{#1}}										% Terminal object
\newcommand{\toone}[1]{!_{#1}}												% Morphism to the terminal object
\newcommand{\comma}[2]{(#1 \mathbin{\downarrow} #2)}						% Comma category
\newcommand{\incl}{\hookrightarrow}            								% Inclusion arrow
\newcommand{\inclby}[1]{\xhookrightarrow{#1}}            					% Inclusion labeled arrow
\newcommand{\lcni}{\hookleftarrow}            								% Inclusion left arrow
\newcommand{\lcniby}[1]{\xhookleftarrow{#1}}            					% Inclusion left labeled arrow
\newcommand{\spn}[5]{#1 \otby{#2} #3 \toby{#4} #5}							% Span
\newcommand{\inclspn}[3]{#1 \lcni #2 \incl #3}								% Span with inclusions
\newcommand{\cospn}[5]{#1 \toby{#2} #3 \otby{#4} #5}						% Cospan
\newcommand{\inclcospn}[3]{#1 \incl #2 \lcni #3}							% Cospan with inclusions
\newcommand{\partialincl}[2]{#1 \lhook\joinrel\rightharpoonup #2}			% Partial monomorphism
\newcommand{\linearrule}[3]{#1 \lcniby{i_{#1}} #2 \inclby{i_{#3}} #3}		% Linear rule
\newcommand{\simplifiedrule}[2]{\linearrule{#1}{#1 \cap #2}{#2}}			% Simplified rule
\newcommand{\match}[3]{#1\from #2\incl #3}									% Monic match
\newcommand{\directderivation}[2]{\Rightarrow^{#1,#2}}						% Direct derivation
\newcommand{\lgraph}[1]{#1\mhyphen\linebreak[0]\scat{Graph}}				% Category of arc-labeled graphs
\newcommand{\sgraph}{\lgraph{\Sigma}}										% Category of graphs arc-labeled on \Sigma
\newcommand{\dims}{\mathbb{D}}												% Set of dimensions
\newcommand{\plustwo}[1]{#1_{+2}}											% Set of pairs (i,j) s.t. i+2<=j
\newcommand{\words}{\mathbb{W}}												% Set of words
\newcommand{\intset}[2]{#1..#2}												% Set of integers betwen two bounds
\newcommand{\qer}[2]{#1/\mathord{#2}}   									% Quotient by equivalence relation
\newcommand{\pathto}{\rightsquigarrow}										% Path
\newcommand{\lpathto}[1]{\overset{#1}{\rightsquigarrow}}					% Labeled path
\newcommand{\relabelingset}{\mathfrak{R}}									% Relabeling set
\newcommand{\embeddingfunctor}[1]{\mathbb{E}_{#1}}							% Embedding functor
\newcommand{\embeddingfunctordef}[3]{%										% Definition of the embedding functor
	\embeddingfunctor{#1}\from\lgraph{#2}\to\lgraph{#3}%
}
\newcommand{\wordsebdfunctor}{\embeddingfunctor{\dims}}						% Embedding functor for words
\newcommand{\wordsebdfunctordef}{%											% Definition of the embedding functor for words
	\embeddingfunctordef{\dims}{\words}{(\words \times \dims)}%
}
\newcommand{\projectingfunctor}[1]{\pi_{#1}}								% Projecting functor
\newcommand{\projectingfunctordef}[3]{%										% Definition of the projecting functor
	\projectingfunctor{#1}\from\lgraph{#2}\to\lgraph{#3}%
}
\newcommand{\dimsprojfunctor}{\projectingfunctor{\dims}}					% Projecting functor on dimensions
\newcommand{\dimsprojfunctordef}{%											% Definition of the projecting functor on dimensions
	\projectingfunctordef{\dims}{(\words \times \dims)}{\dims}%
}					
\newcommand{\wordsprojfunctor}{\projectingfunctor{\words}}					% Projecting functor on words
\newcommand{\wordsprojfunctordef}{%											% Definition of the projecting functor on words
	\projectingfunctordef{\dims}{(\words \times \dims)}{\words}%
}	

\DeclareMathOperator{\ob}{\mathcal{O}}  									% Objects
\DeclareMathOperator{\inc}{\mathbf{i}}  									% Incidence predicat
\DeclareMathOperator{\edg}{\mathbf{e}}  									% Edge prodicat
\DeclareMathOperator{\nod}{\mathbf{n}}  									% Node prodicat
\DeclareMathOperator{\lab}{\mathbf{l}}  									% Label prodicat

% G/O maps
\newcommand{\incidentarcs}[2]{\mathtt{I}_{#1}(#2)}   						% Incident arcs constraint
\newcommand{\nonorientation}[2]{\mathtt{O}_{#1}(#2)}   						% Non orientations constraint
\newcommand{\cycles}[2]{\mathtt{C}_{#1}(#2)}   								% Incident arcs constraint
\newcommand{\azero}{%														% 0-arc
	\textcolor{alpha0}{%											
		0%
	}%
}					
\newcommand{\aone}{%														% 1-arc
	\textcolor{alpha1}{%
		1%
	}%
}					
\newcommand{\atwo}{%														% 2-arc
	\textcolor{alpha2}{%
		2%
	}%
}
\newcommand{\athree}{%														% 3-arc
	\textcolor{alpha3}{%
		3%
	}%
}




% restore \coprod
\DeclareRobustCommand{\coprod}{\mathop{\text{\fakecoprod}}}
\newcommand{\fakecoprod}{%
  \sbox0{$\prod$}%
  \smash{\raisebox{\dimexpr.9625\depth-\dp0}{\scalebox{1}[-1]{$\prod$}}}%
  \vphantom{$\prod$}%
}

% Algorithmes
\newcommand\commfont[1]{\scriptsize\ttfamily\textcolor{blue}{#1}}
\SetCommentSty{commfont}

% Colors
\definecolor{Prune}{RGB}{99,0,60}
\definecolor{myorange}{RGB}{251, 131, 76}
\definecolor{mygreen}{RGB}{196, 251, 76}
\definecolor{mypurple}{RGB}{131, 76, 251}
\definecolor{mypink}{RGB}{251, 76, 196}
\definecolor{mybrown}{RGB}{167, 93, 71}
\definecolor{myyellow}{RGB}{251, 219, 76}
\definecolor{myblue}{RGB}{76, 196, 251}
\definecolor{mylorange}{RGB}{253, 183, 151}
\definecolor{mylgreen}{RGB}{221, 253, 151}
\definecolor{mylpurple}{RGB}{183, 151, 253}
\definecolor{mylpink}{RGB}{253, 151, 221}
\definecolor{mylbrown}{RGB}{197, 136, 118}
\definecolor{mylyellow}{RGB}{253, 234, 151}
\definecolor{mylblue}{RGB}{151, 221, 253}
\newcommand{\opSet}{
	\{%
		\textcolor{myorange}{\CIRCLE},
		\textcolor{mypurple}{\CIRCLE}
	\}%
}
\newcommand{\opbSet}{
	\{%
		\textcolor{myorange}{\CIRCLE},
		\textcolor{mypurple}{\CIRCLE},
		\textcolor{myblue}{\CIRCLE}
	\}%
}
\definecolor{alpha0}{RGB}{0, 0, 0}
\definecolor{alpha1}{RGB}{255, 0, 0}
\definecolor{alpha2}{RGB}{0, 0, 255}
\definecolor{alpha3}{RGB}{0, 128, 0}
\newcommand{\orangei}{%
	\textcolor{myorange}{%
		i%
	}%
}
\newcommand{\purplej}{%
	\textcolor{mypurple}{%
		j%
	}%
}

% Misc
\newcommand{\ligne}[1]
{\begin{center}
    \color{#1}
    \rule{\textwidth}{0.1pt}
    \end{center}
}

\newcommand*\circled[1]{\tikz[baseline=(char.base)]{
            \node[shape=circle,draw,fill=black,inner sep=0.7pt] (char) {$\textcolor[rgb]{1,1,1}{#1}$};}}

%\newcommand*\mycircled[1]{\tikz[baseline=(char.base)]{
%            \node[shape=circle,draw,inner sep=0.4pt] (char) {#1};}}

%% On commence par une page impaire quand on change le style de numérotation de pages 
\let\oldpagenumbering\pagenumbering
\renewcommand{\pagenumbering}[1]{%
	\cleardoublepage
	\oldpagenumbering{#1}
}


% inline arcs and paths
\newcommand{\arc}{\rule[1.5pt]{7pt}{1.5pt}}
\newcommand{\zeroarc}{\textcolor{alpha0}{\arc}}
\newcommand{\onearc}{\textcolor{alpha1}{\arc}}
\newcommand{\twoarc}{\textcolor{alpha2}{\arc}}
\newcommand{\threearc}{\textcolor{alpha3}{\arc}}
\newcommand{\larc}[1]{%
    \raisebox{-.9pt}{%
        $\, \mathbin{\bullet\mkern-3mu{\overset{#1}-}\mkern-3mu\bullet} \,$}
}
\newcommand{\ijpath}[4]{%
    \raisebox{-.9pt}{%
        $\, \mathbin{%
        \bullet
        \mkern-3mu{\overset{#1}-}\mkern-7mu\bullet
        \mkern-7mu{\overset{#2}-}\mkern-7mu\bullet
        \mkern-7mu{\overset{#3}-}\mkern-7mu\bullet
        \mkern-7mu{\overset{#4}-}\mkern-3mu\bullet
        } \,$}
}

% code Jerboa
\lstdefinestyle{Jerboa}{
	language=Java,
	basicstyle=\footnotesize,
	frame = trbl,
	framesep = 1mm,
	framerule = 1pt,
	frameround = tttt,
	rulecolor = \color[rgb]{0.9,0.9,1},
	belowskip =3mm,
	xrightmargin = 2mm,
	xleftmargin = 2mm,
	%columns=fullflexible,
	showstringspaces=false,
	extendedchars=true,
	breaklines=true,
    breakatwhitespace=true,
	showtabs=false,
	showspaces=false,
	identifierstyle=\ttfamily,
	keywordstyle=\color[rgb]{0,0,0.75}\bfseries,
	commentstyle=\color[rgb]{0.133,0.545,0.133},
	stringstyle=\color[rgb]{0.627,0.126,0.941},
	backgroundcolor =\color[rgb]{0.98,0.98,1},
    emph={_position,Point3,@0,@1,@2,@3},
}